%%%%%%%%%%%%%%%%%%%%%%%%%%%%%%%%%%%%%%%%%%%%%%%%%%%%%%%%%%%%%%%%%%%%%%
%!TEX root = diss.tex
\chapter{Introduction}
\label{ch:Introduction}

This document provides a quick set of instructions for using the
\verb+ubcdiss+ class to write a dissertation in \LaTeX.

It is far beyond the scope of this document to provide an introduction
to LaTeX.  The classic reference for learning \LaTeX is Leslie Lamport's
book~\cite{lamport-1994-ladps}.
\href{http://www.andy-roberts.net/misc/latex/}{Andy Roberts' online
\LaTeX\ tutorials}\footnote{%
    \url{http://www.andy-roberts.net/misc/latex/}}
also seem to be excellent.  Ignore their examples on using the
\verb+tabular+ environment to typset tables: the \verb+booktabs+
environment is far more preferable.

It is generally a good idea to put each chapter in a separate file.
We discuss organizational issues further in
\autoref{sec:SuggestedThesisOrganization}

It is also generally a good idea to use \emph{version control}
when writing your dissertation.  We discuss version control further
in \autoref{sec:DissertationVersionControl}.

We should perhaps make reference to \ac{ANOVA} here.

\begin{itemize}
\item 66 cpl; adjusting textwidth using the geometry package
\item how to make tables with booktabs, figures
\item the chngpage package and adjustwidth
\item bibtex, natbib (citep vs citet), and .bst files
\item includegraphics
\end{itemize}

\section{Suggested Thesis Organization}
\label{sec:SuggestedThesisOrganization}

The recommended best practice for organizing large documents
in \LaTeX\ is to 

in \label{sec:DissertationVersionControl}.

\endinput

Any text after an \endinput is ignored.
You could put scraps here or things in progress.
