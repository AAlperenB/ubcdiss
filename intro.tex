%%%%%%%%%%%%%%%%%%%%%%%%%%%%%%%%%%%%%%%%%%%%%%%%%%%%%%%%%%%%%%%%%%%%%%
%!TEX root = diss.tex
\chapter{Introduction}
\label{ch:Introduction}

It is far beyond the scope of this document to provide an introduction
to LaTeX. 
The classic reference for learning LaTeX is Leslie Lamport's
book~\cite{lamport-1994-ladps}.
\href{http://www.andy-roberts.net/misc/latex/}{Andy Roberts' online
\LaTeX\ tutorials}\footnote{%
    Found at \url{http://www.andy-roberts.net/misc/latex/}}
are an excellent alternative.

It is generally a good idea to put each chapter in a separate file.
We discuss organizational issues further in
\autoref{sec:SuggestedThesisOrganization}

It is also generally a good idea to use \emph{version control}
when writing your dissertation.  We discuss version control further
in \autoref{sec:DissertationVersionControl}.

We should perhaps make reference to \ac{ANOVA} here.

\endinput

Any text after an \endinput is ignored.
You could put scraps here or things in progress.
