%%%%%%%%%%%%%%%%%%%%%%%%%%%%%%%%%%%%%%%%%%%%%%%%%%%%%%%%%%%%%%%%%%%%%%
% Template for a UBC-compliant dissertation
% At the minimum, you will need to change the information found
% after the "Document meta-data"
%
%!TEX TS-program = pdflatex
%!TEX encoding = UTF-8 Unicode

%% The ubcdiss class provides several options:
%%   fogscopy
%%       set parameters to exactly how FoGS specifies
%%         * single-sided
%%         * page-numbering starts from title page
%%         * the lists of figures and tables have each entry prefixed
%%           with 'Figure' or 'Table'
%%       This can be tested by `\iffogscopy ... \else ... \fi'
%%   10pt, 11pt, 12pt
%%       set default font size
%%   oneside, twoside
%%       whether to format for single-sided or double-sided printing
%%   balanced
%%       when double-sided, ensure page content is centred
%%       rather than slightly offset (the default)
%%   singlespacing, onehalfspacing, doublespacing
%%       set default inter-line text spacing; the ubcdiss class
%%       provides \textspacing to revert to this configured spacing
%%   draft
%%       disable more intensive processing, such as including
%%       graphics, etc.
%%

% For submission to FoGS
\documentclass[fogscopy,onehalfspacing,11pt]{ubcdiss}

% For your own copies (looks nicer)
% \documentclass[balanced,twoside,11pt]{ubcdiss}

%%%%%%%%%%%%%%%%%%%%%%%%%%%%%%%%%%%%%%%%%%%%%%%%%%%%%%%%%%%%%%%%%%%%%%
%%%%%%%%%%%%%%%%%%%%%%%%%%%%%%%%%%%%%%%%%%%%%%%%%%%%%%%%%%%%%%%%%%%%%%
%%
%% FONTS:
%% 
%% The defaults below configures Times Roman for the serif font,
%% Helvetica for the sans serif font, and Courier for the
%% typewriter-style font.  Configuring fonts can be time
%% consuming; we recommend skipping to END FONTS!
%% 
%% If you're feeling brave, have lots of time, and wish to use one
%% your platform's native fonts, see the commented out bits below for
%% XeTeX/XeLaTeX.  This is not for the faint at heart. 
%% (And shouldn't you be writing? :-)
%%

%% NFSS font specification (New Font Selection Scheme)
\usepackage{times,mathptmx,courier}
\usepackage[scaled=.92]{helvet}

%% Math or theory people may want to include the handy AMS macros
%\usepackage{amssymb}
%\usepackage{amsmath}
%\usepackage{amsfonts}

%% The pifont package provides access to the elements in the dingbat font.   
%% Use \ding{##} for a particular dingbat (see p7 of psnfss2e.pdf)
%%   Useful:
%%     51,52 different forms of a checkmark
%%     54,55,56 different forms of a cross (saltyre)
%%     172-181 are 1-10 in open circle (serif)
%%     182-191 are 1-10 black circle (serif)
%%     192-201 are 1-10 in open circle (sans serif)
%%     202-211 are 1-10 in black circle (sans serif)
%% \begin{dinglist}{##}\item... or dingautolist (which auto-increments)
%% to create a bullet list with the provided character.
\usepackage{pifont}

%%%%%%%%%%%%%%%%%%%%%%%%%%%%%%%%%%%%%%%%%%%%%%%%%%%%%%%%%%%%%%%%%%%%%%
%% Configure fonts for XeTeX / XeLaTeX using the fontspec package.
%% Be sure to check out the fontspec documentation.
%\usepackage{fontspec,xltxtra,xunicode}	% required
%\defaultfontfeatures{Mapping=tex-text}	% recommended
%% Minion Pro and Myriad Pro are shipped with some versions of
%% Adobe Reader.  Adobe representatives have commented that these
%% fonts can be used outside of Adobe Reader.
%\setromanfont[Numbers=OldStyle]{Minion Pro}
%\setsansfont[Numbers=OldStyle,Scale=MatchLowercase]{Myriad Pro}
%\setmonofont[Scale=MatchLowercase]{Andale Mono}

%% Other alternatives:
%\setromanfont[Mapping=tex-text]{Adobe Caslon}
%\setsansfont[Scale=MatchLowercase]{Gill Sans}
%\setsansfont[Scale=MatchLowercase,Mapping=tex-text]{Futura}
%\setmonofont[Scale=MatchLowercase]{Andale Mono}
%\newfontfamily{\SYM}[Scale=0.9]{Zapf Dingbats}
%% END FONTS
%%%%%%%%%%%%%%%%%%%%%%%%%%%%%%%%%%%%%%%%%%%%%%%%%%%%%%%%%%%%%%%%%%%%%%
%%%%%%%%%%%%%%%%%%%%%%%%%%%%%%%%%%%%%%%%%%%%%%%%%%%%%%%%%%%%%%%%%%%%%%



%%%%%%%%%%%%%%%%%%%%%%%%%%%%%%%%%%%%%%%%%%%%%%%%%%%%%%%%%%%%%%%%%%%%%%
%%%%%%%%%%%%%%%%%%%%%%%%%%%%%%%%%%%%%%%%%%%%%%%%%%%%%%%%%%%%%%%%%%%%%%
%%
%% Recommended packages
%%
\usepackage{checkend}	% better error messages on left-open environments
\usepackage{graphicx}	% for incorporating external images

%% booktabs: provides some special commands for typesetting tables as used
%% in excellent journals.  Ignore the examples in the Lamport book!
\usepackage{booktabs}

%% listings: useful support for including source code listings, with
%% optional special keyword formatting.  The \lstset{} causes
%% the text to be typeset in a smaller sans serif font, with
%% proportional spacing.
\usepackage{listings}
\lstset{basicstyle=\sffamily\scriptsize,showstringspaces=false,fontadjust}

%% The acronym package provides support for defining acronyms, providing
%% their expansion when first used, and building glossaries.  See the
%% example in glossary.tex and the example usage throughout the example
%% document.
%% NOTE: to use \MakeTextLowercase in the \acsfont command below,
%%   we *must* use the `nohyperlinks' option -- it causes errors with
%%   hyperref otherwise.  See Section 5.2 in the ``LaTeX 2e for Class
%%   and Package Writers Guide'' (clsguide.pdf) for details.
\usepackage[printonlyused,nohyperlinks]{acronym}
%% The ubcdiss.cls loads the `textcase' package which provides commands
%% for upper-casing and lower-casing text.  The following causes
%% the acronym package to typeset acronyms in small-caps
%% as recommended by Bringhurst.
\renewcommand{\acsfont}[1]{{\scshape \MakeTextLowercase{#1}}}

%% color: add support for expressing colour models.  Grey can be used
%% to great effect to emphasize other parts of a graphic or text.
%% For an excellent set of examples, see Tufte's "Visual Display of
%% Quantitative Information" or "Envisioning Information".
\usepackage{color}
\definecolor{greytext}{gray}{0.5}

%% comment: provides a new {comment} environment: all text inside the
%% environment is ignored.
%%   \begin{comment} ignored text ... \end{comment}
\usepackage{comment}

%% The natbib package provides more sophisticated citing commands
%% such as \citeauthor{} to provide the author names of a work,
%% \citet{} to produce an author-and-reference citation,
%% \citep{} to produce a parenthetical citation.
%% We use \citeeg{} to provide examples
\usepackage[numbers,sort&compress]{natbib}
\newcommand{\citeeg}[1]{\citep[e.g.,][]{#1}}

%% The titlesec package provides commands to vary how chapter and
%% section titles are typeset.  The following uses more compact
%% spacings above and below the title.  The titleformat that follow
%% ensure chapter/section titles are set in singlespace.
\usepackage[compact]{titlesec}
\titleformat*{\section}{\singlespacing\raggedright\bfseries\Large}
\titleformat*{\subsection}{\singlespacing\raggedright\bfseries\large}
\titleformat*{\subsubsection}{\singlespacing\raggedright\bfseries}
\titleformat*{\paragraph}{\singlespacing\raggedright\itshape}

%% The caption package provides support for varying how table and
%% figure captions are typeset.
\usepackage[format=hang,indention=-1cm,labelfont={bf},margin=1em]{caption}

%% url: for typesetting URLs and smart(er) hyphenation.
%% \url{http://...} 
\usepackage{url}
\urlstyle{sf}	% typeset urls in sans-serif


%%%%%%%%%%%%%%%%%%%%%%%%%%%%%%%%%%%%%%%%%%%%%%%%%%%%%%%%%%%%%%%%%%%%%%
%%%%%%%%%%%%%%%%%%%%%%%%%%%%%%%%%%%%%%%%%%%%%%%%%%%%%%%%%%%%%%%%%%%%%%
%%
%% Possibly useful packages: you may need to explicitly install
%% these from CTAN if they aren't part of your distribution;
%% teTeX seems to ship with a smaller base than MikTeX and MacTeX.
%%
%\usepackage{pdfpages}	% insert pages from other PDF files
%\usepackage{longtable}	% provide tables spanning multiple pages
%\usepackage{chngpage}	% support changing the page widths on demand
%\usepackage{tabularx}	% an enhanced tabular environment

%% enumitem: support pausing and resuming enumerate environments.
%\usepackage{enumitem}

%% rotating: provides two environments, sidewaystable and sidewaysfigure,
%% for typesetting tables and figures in landscape mode.  
%\usepackage{rotating}

%% subfig: provides for including subfigures within a figure,
%% and includes being able to separately reference the subfigures.
%\usepackage{subfig}

%% ragged2e: provides several new new commands \Centering, \RaggedLeft,
%% \RaggedRight and \justifying and new environments Center, FlushLeft,
%% FlushRight and justify, which set ragged text and are easily
%% configurable to allow hyphenation.
%\usepackage{ragged2e}

%% The ulem package provides a \sout{} for striking out text and
%% \xout for crossing out text.  The normalem and normalbf are
%% necessary as the package messes with the emphasis and bold fonts
%% otherwise.
%\usepackage[normalem,normalbf]{ulem}    % for \sout

%%%%%%%%%%%%%%%%%%%%%%%%%%%%%%%%%%%%%%%%%%%%%%%%%%%%%%%%%%%%%%%%%%%%%%
%% HYPERREF:
%% The hyperref package provides for embedding hyperlinks into your
%% document.  By default the table of contents, references, citations,
%% and footnotes are hyperlinked.
%%
%% Hyperref provides a very handy command for doing cross-references:
%% \autoref{}.  This is similar to \ref{} and \pageref{} except that
%% it automagically puts in the *type* of reference.  For example,
%% referencing a figure's label will put the text `Figure 3.4'.
%% And the text will be hyperlinked to the appropriate place in the
%% document.
%%
%% Generally hyperref should appear after most other packages

%% The following puts hyperlinks in very faint grey boxes.
%% The `pagebackref' causes the references in the bibliography to have
%% back-references to the citing page; `backref' puts the citing section
%% number.  See further below for other examples of using hyperref.
%% 2009/12/09: now use `linktocpage' (Jacek Kisynski): FoGS now prefers
%%   that the ToC, LoF, LoT place the hyperlink on the page number,
%%   rather than the entry text.
\usepackage[bookmarks,bookmarksnumbered,%
    citebordercolor={0.8 0.8 0.8},filebordercolor={0.8 0.8 0.8},%
    linkbordercolor={0.8 0.8 0.8},pagebordercolor={0.8 0.8 0.8},%
    urlbordercolor={0.8 0.8 0.8},%
    pagebackref,linktocpage%
    ]{hyperref}
%% The following change how the the back-references text is typeset in a
%% bibliography when `backref' or `pagebackref' are used
\renewcommand\backrefpagesname{\(\rightarrow\) pages}
\renewcommand\backref{\textcolor{greytext} \backrefpagesname\ }

%% The following uses most defaults, which causes hyperlinks to be
%% surrounded by colourful boxes; the colours are only visible in
%% PDFs and don't show up when printed:
%\usepackage[bookmarks,bookmarksnumbered]{hyperref}

%% The following disables the colourful boxes around hyperlinks.
%\usepackage[bookmarks,bookmarksnumbered,pdfborder={0 0 0}]{hyperref}

%% The following disables all hyperlinking, but still enabled use of
%% \autoref{}
%\usepackage[draft]{hyperref}

%% The following commands causes chapter and section references to
%% uppercase the part name.
\renewcommand{\chapterautorefname}{Chapter}
\renewcommand{\sectionautorefname}{Section}
\renewcommand{\subsectionautorefname}{Section}
\renewcommand{\subsubsectionautorefname}{Section}


%%%%%%%%%%%%%%%%%%%%%%%%%%%%%%%%%%%%%%%%%%%%%%%%%%%%%%%%%%%%%%%%%%%%%%
%%%%%%%%%%%%%%%%%%%%%%%%%%%%%%%%%%%%%%%%%%%%%%%%%%%%%%%%%%%%%%%%%%%%%%
%%
%% Some special settings that controls how text is typeset
%%
% \raggedbottom		% pages don't have to line up nicely on the last line
% \sloppy		% be a bit more relaxed in inter-word spacing
% \clubpenalty=10000	% try harder to avoid orphans
% \widowpenalty=10000	% try harder to avoid widows
% \tolerance=1000

%% And include some of our own useful macros
%% The following is a directive for TeXShop to indicate the main file
%%!TEX root = template.tex
%
% This file provides examples of some useful macros for typesetting
% dissertations.  None of the macros defined here are necessary beyond
% for the template.tex file: feel free to change, remove, and add
% your own definitions.
%
% We recommend that you define macros to separate the semantics
% of the things you write from how they are presented.  For example,
% you'll see definitions below for a macro \filename{}: by using
% \filename{} consistently in the text, we can change how filenames
% are typeset simply by changing the definition of \filename{} in
% this file.
% 

\newcommand{\NA}{\textsc{n/a}}
\newcommand{\eg}{e.g.,\ }
\newcommand{\ie}{i.e.,\ }

% Some useful macros for typesetting LaTeX terms
\newcommand{\BibTeX}{Bib\TeX}
\newcommand{\etal}{\emph{et al}}
\newcommand{\file}[1]{\texttt{#1}}
\newcommand{\class}[1]{\texttt{#1}}
\newcommand{\package}[1]{\href{http://www.ctan.org/macros/latex/contrib/#1}{\texttt{#1}}}
\newcommand{\packagemisc}[1]{\href{http://www.ctan.org/macros/latex/contrib/misc/#1.sty}{\texttt{#1}}}
\newcommand{\env}[1]{\texttt{#1}}

% Define a command \doi{} to typeset a digital object identifier (DOI).
% The first version requires the `url' package (which is loaded by
% ubcdiss.cls).  The commented-out second variant does not.
\DeclareUrlCommand\DOI{}
\newcommand{\doi}[1]{\href{http://dx.doi.org/#1}{\DOI{doi:#1}}}
%\newcommand{\doi}[1]{\href{http://dx.doi.org/#1}{doi:#1}}

% Useful macro to reference an online document with a hyperlink
% as well with the URL explicitly listed in a footnote
% #1: the URL
% #2: the anchoring text
\newcommand{\webref}[2]{\href{#1}{#2}\footnote{\url{#1}}}

% epigraph is a nice environment for typesetting quotations
\makeatletter
\newenvironment{epigraph}{%
	\begin{flushright}
	\begin{minipage}{\columnwidth-0.75in}
	\begin{flushright}
	\@ifundefined{singlespacing}{}{\singlespacing}%
    }{
	\end{flushright}
	\end{minipage}
	\end{flushright}}
\makeatother

% \FIXME{} is a useful macro for noting things needing to be changed.
% The following definition will also output a warning to the console
\newcommand{\FIXME}[1]{\typeout{**FIXME** #1}\textbf{[FIXME: #1]}}

% END


%%%%%%%%%%%%%%%%%%%%%%%%%%%%%%%%%%%%%%%%%%%%%%%%%%%%%%%%%%%%%%%%%%%%%%
%%%%%%%%%%%%%%%%%%%%%%%%%%%%%%%%%%%%%%%%%%%%%%%%%%%%%%%%%%%%%%%%%%%%%%
%%
%% Document meta-data: be sure to also change the \hypersetup information
%%

\title{On the Use of the \texttt{ubcdiss} Template}
%\subtitle{If you want a subtitle}

\author{Johnny Canuck}
\previousdegree{B. Basket Weaving, University of Illustrious Arts, 1991}
\previousdegree{M. Silly Walks, Another University, 1994}

% What is this dissertation for?
\degreetitle{Doctor of Philosophy}

\institution{The University Of British Columbia}
\campus{Vancouver}

\faculty{The Faculty of Graduate Studies}
\department{Basket Weaving}
\submissionmonth{April}
\submissionyear{2192}

%% hyperref package provides support for embedding meta-data in .PDF
%% files
\hypersetup{
  pdftitle={Change this title!  (DRAFT: \today)},
  pdfauthor={Johnny Canuck},
  pdfkeywords={Your keywords here}
}

%%%%%%%%%%%%%%%%%%%%%%%%%%%%%%%%%%%%%%%%%%%%%%%%%%%%%%%%%%%%%%%%%%%%%%
%%%%%%%%%%%%%%%%%%%%%%%%%%%%%%%%%%%%%%%%%%%%%%%%%%%%%%%%%%%%%%%%%%%%%%
%% 
%% The document content
%%

%% LaTeX's \includeonly commands causes any uses of \include{} to only
%% include files that are in the list.  This is helpful to produce
%% subsets of your thesis (e.g., for committee members who want to see
%% the dissertation chapter by chapter).  It also saves time by 
%% avoiding reprocessing the entire file.
%\includeonly{intro,conclusions}
%\includeonly{discussion}

\begin{document}

%%%%%%%%%%%%%%%%%%%%%%%%%%%%%%%%%%%%%%%%%%%%%%%%%%
%% From Thesis Components: Tradtional Thesis
%% <http://www.grad.ubc.ca/current-students/dissertation-thesis-preparation/order-components>

% Preliminary Pages (numbered in lower case Roman numerals)
%    1. Title page (mandatory)
\maketitle

%    2. Abstract (mandatory - maximum 350 words)
%% The following is a directive for TeXShop to indicate the main file
%%!TEX root = diss.tex

\chapter{Abstract}

This document provides brief instructions for using the \class{ubcdiss}
class to write a \acs{UBC}-conformant dissertation in \LaTeX.  This
document is itself written using the \class{ubcdiss} class and is
intended to serve as an example of writing a dissertation in \LaTeX.
This document has embedded \acp{URL} and is intended to be viewed
using a computer-based \ac{PDF} reader.

% Note: these must also be defined in glossary.tex
\newacro{UBC}{The University of British Columbia}
\newacro{FoGS}[FoGS]{Faculty of Graduate Studies}
\newacro{PDF}{Portable Document Format}
\newacro{URL}{Uniform Resource Locator}

Note: Abstracts should generally try to avoid using acronyms.  If you
must, use the \verb+acronym+ package's \verb+\newacro+ to define
local versions here; these definitions will be overwritten by those
in \file{glossary.tex}.  See the \file{glossary.tex} file for
examples of defining elements for a glossary.
For example, the source for this page added the following definition:
\begin{verbatim}
    \newacro{UBC}{The University of British Columbia}
    \newacro{FoGS}{Faculty of Graduate Studies}
    \newacro{PDF}{Portable Document Format}
    \newacro{URL}{Uniform Resource Locator}
\end{verbatim}
so as to refer to `\ac{UBC}' and `\ac{FoGS}'.

Note: at \ac{UBC}, both the \ac{FoGS} Ph.D. defence programme and the
Library's online submission system require that abstracts to be 350
words or less.

% Embed version information inline -- you should remove this from your
% dissertation
\vfill
\begin{center}
\begin{sf}
\fbox{Revision: ubcdiss.cls r12
}
\end{sf}
\end{center}

\cleardoublepage

%    3. Preface
%% The following is a directive for TeXShop to indicate the main file
%%!TEX root = diss.tex

\chapter{Preface}

At \ac{UBC}, a preface may be required.  Be sure to check the
\ac{FoGS} guidelines as they may have specific content to be included.

\cleardoublepage

%    4. Table of contents (mandatory - list all items in the preliminary pages
%    starting with the abstract, followed by chapter headings and
%    subheadings, bibliographies and appendices)
\tableofcontents
\cleardoublepage	% required by tocloft package

%    5. List of tables (mandatory if thesis has tables)
\listoftables
\cleardoublepage	% required by tocloft package

%    6. List of figures (mandatory if thesis has figures)
\listoffigures
\cleardoublepage	% required by tocloft package

%    7. List of illustrations (mandatory if thesis has illustrations)
%    8. Lists of symbols, abbreviations or other (optional)

%    9. Glossary (optional)
%% The following is a directive for TeXShop to indicate the main file
%%!TEX root = diss.tex

\chapter{Glossary}

This glossary uses the handy \latexpackage{acroynym} package to automatically
maintain the glossary.  It uses the package's \texttt{printonlyused}
option to include only those acronyms explicitly referenced in the
\LaTeX\ source.

% use \acrodef to define an acronym, but no listing
\acrodef{UI}{user interface}
\acrodef{UBC}{University of British Columbia}

% The acronym environment will typeset only those acronyms that were
% *actually used* in the course of the document
\begin{acronym}[ANOVA]
\acro{ANOVA}[ANOVA]{Analysis of Variance\acroextra{, a set of
  statistical techniques to identify sources of variability between groups}}
\acro{API}{application programming interface}
\acro{CTAN}{\acroextra{The }Common \TeX\ Archive Network}
\acro{DOI}{Document Object Identifier\acroextra{ (see
    \url{http://doi.org})}}
\acro{FoGS}[FoGS]{The Faculty of Graduate Studies}
\acro{PDF}{Portable Document Format}
\acro{RCS}[RCS]{Revision control system\acroextra{, a software
    tool for tracking changes to a set of files}}
\acro{TLX}[TLX]{Task Load Index\acroextra{, an instrument for gauging
  the subjective mental workload experienced by a human in performing
  a task}}
\acro{UML}{Unified Modelling Language\acroextra{, a visual language
    for modelling the structure of software artefacts}}
\acro{URL}{Unique Resource Locator\acroextra{, used to describe a
    means for obtaining some resource on the world wide web}}
\acro{W3C}[W3C]{\acroextra{the }World Wide Web Consortium\acroextra{,
    the standards body for web technologies}}
\acro{XML}{Extensible Markup Language}
\end{acronym}

% You can also use \newacro{}{} to only define acronyms
% but without explictly creating a glossary
% 
% \newacro{ANOVA}[ANOVA]{Analysis of Variance\acroextra{, a set of
%   statistical techniques to identify sources of variability between groups.}}
% \newacro{API}[API]{application programming interface}
% \newacro{GOMS}[GOMS]{Goals, Operators, Methods, and Selection\acroextra{,
%   a framework for usability analysis.}}
% \newacro{TLX}[TLX]{Task Load Index\acroextra{, an instrument for gauging
%   the subjective mental workload experienced by a human in performing
%   a task.}}
% \newacro{UI}[UI]{user interface}
% \newacro{UML}[UML]{Unified Modelling Language}
% \newacro{W3C}[W3C]{World Wide Web Consortium}
% \newacro{XML}[XML]{Extensible Markup Language}
	% always input, since other macros may rely on it

\textspacing		% begin one-half or double spacing

%   10. Acknowledgements (optional)
%!TEX root = diss.tex
\chapter{Acknowledgments}

Thank those people who helped you. 
Don't forget your parents or loved ones.
You may wish to include funding bodies.


%   11. Dedication (optional)

% Body of Thesis (not all sections may apply)
\mainmatter

\acresetall	% reset all acronyms used so far

%    1. Introduction
%%%%%%%%%%%%%%%%%%%%%%%%%%%%%%%%%%%%%%%%%%%%%%%%%%%%%%%%%%%%%%%%%%%%%%
%!TEX root = diss.tex
\chapter{Introduction}
\label{ch:Introduction}

This document provides a quick set of instructions for using the
\verb+ubcdiss+ class to write a dissertation in \LaTeX.

It is far beyond the scope of this document to provide an introduction
to LaTeX.  The classic reference for learning \LaTeX is Leslie Lamport's
book~\cite{lamport-1994-ladps}.
\href{http://www.andy-roberts.net/misc/latex/}{Andy Roberts' online
\LaTeX\ tutorials}\footnote{%
    \url{http://www.andy-roberts.net/misc/latex/}}
also seem to be excellent.  Ignore their examples on using the
\verb+tabular+ environment to typset tables: the \verb+booktabs+
environment is far more preferable.

It is generally a good idea to put each chapter in a separate file.
We discuss organizational issues further in
\autoref{sec:SuggestedThesisOrganization}

It is also generally a good idea to use \emph{version control}
when writing your dissertation.  We discuss version control further
in \autoref{sec:DissertationVersionControl}.

We should perhaps make reference to \ac{ANOVA} here.

\begin{itemize}
\item 66 cpl; adjusting textwidth using the geometry package
\item how to make tables with booktabs, figures
\item the chngpage package and adjustwidth
\item bibtex, natbib (citep vs citet), and .bst files
\item includegraphics
\end{itemize}

\section{Suggested Thesis Organization}
\label{sec:SuggestedThesisOrganization}

The recommended best practice for organizing large documents
in \LaTeX\ is to 

in \label{sec:DissertationVersionControl}.

\endinput

Any text after an \endinput is ignored.
You could put scraps here or things in progress.


%    2. Main body
% Generally recommended to put each chapter into a separate file
%\include{relatedwork}
%\include{model}
%\include{impl}
%\include{discussion}
%\include{conclusions}

%    3. Notes
%    4. Footnotes

%    5. Bibliography
\begin{singlespace}
\raggedright
\bibliographystyle{abbrvnat}
\bibliography{biblio}
\end{singlespace}

\appendix
%    6. Appendices (including copies of all required UBC Research
%       Ethics Board's Certificates of Approval)
%\include{reb-coa}	% pdfpages is useful here

\backmatter
%    7. Index
% See the makeindex package: the following page provides a quick overview
% <http://www.image.ufl.edu/help/latex/latex_indexes.shtml>


\end{document}
